%%%%%%%%%%%%%%%%%%%%%%%%%%%%%%%%%%%%%%%%%
%  My documentation report
%  Objetive: Explain what I did and how, so someone can continue with the investigation
%
% Important note:
% Chapter heading images should have a 2:1 width:height ratio,
% e.g. 920px width and 460px height.
%
%%%%%%%%%%%%%%%%%%%%%%%%%%%%%%%%%%%%%%%%%


%----------------------------------------------------------------------------------------
%	PACKAGES AND OTHER DOCUMENT CONFIGURATIONS
%----------------------------------------------------------------------------------------

\documentclass[11pt,fleqn]{book} % Default font size and left-justified equations

\usepackage[top=3cm,bottom=3cm,left=3.2cm,right=3.2cm,headsep=10pt,letterpaper]{geometry} % Page margins

\usepackage{xcolor} % Required for specifying colors by name
\definecolor{ocre}{RGB}{52,177,201} % Define the orange color used for highlighting throughout the book


% Font Settings
\usepackage{avant} % Use the Avantgarde font for headings
%\usepackage{times} % Use the Times font for headings
\usepackage{mathptmx} % Use the Adobe Times Roman as the default text font together with math symbols from the Sym­bol, Chancery and Com­puter Modern fonts
\usepackage{microtype} % Slightly tweak font spacing for aesthetics

\usepackage[utf8]{inputenc} % Required for including letters with accents
\usepackage{amsmath,amssymb}
\usepackage[russian]{babel}
\usepackage[T1]{fontenc} % Use 8-bit encoding that has 256 glyphs

 
\urlstyle{same}

% Bibliography
\usepackage[style=alphabetic,sorting=nyt,sortcites=true,autopunct=true,babel=hyphen,hyperref=true,abbreviate=false,backref=true,backend=biber]{biblatex}
\addbibresource{bibliography.bib} % BibTeX bibliography file
\defbibheading{bibempty}{}

\input{structure} % Insert the commands.tex file which contains the majority of the structure behind the template

%----------------------------------------------------------------------------------------
%	defis of new commands
%----------------------------------------------------------------------------------------

\def\R{\mathbb{R}}
\def\N{\mathcal{N}}
\def\t{\theta}
\def\T{\Theta}
\def\Exp{\textit{Exp}}
\def\H{\hat{\theta}}
\def\X{(X_1, \cdots, X_n)}
\def\s{\textit{  }}
\def\DT{\xrightarrow[] {d_\theta}}
\def\d{\xrightarrow[] {d}}
\def\p{\xrightarrow[] {P}}
\def\PPT{\xrightarrow[P_\theta - \textit{п.н}] {P_\theta}}
\def\PT{\xrightarrow[] {P}}

\newcommand{\cvx}{convex}
\begin{document}

%----------------------------------------------------------------------------------------
%	TITLE PAGE
%----------------------------------------------------------------------------------------

\begingroup
\thispagestyle{empty}
\AddToShipoutPicture*{\put(130,30){\includegraphics[scale=0.5]{Verchick-cats-graph-2}}} % Image background
\centering
\vspace*{5cm}
\par\normalfont\fontsize{35}{35}\sffamily\selectfont
\textbf{Прикладная стастистика}\\
{\LARGE  ФИВТ ОСЕНЬ 2019}\par % Book title
\vspace*{1cm}
{\Huge Лекционные записи}\par % Author name
\endgroup

%----------------------------------------------------------------------------------------
%	COPYRIGHT PAGE
%----------------------------------------------------------------------------------------

\newpage
~\vfill
\thispagestyle{empty}

%\noindent Copyright \copyright\ 2014 Andrea Hidalgo\\ % Copyright notice

\noindent \textsc{Summer Research Internship, University of Western Ontario}\\

\noindent \textsc{github.com/LaurethTeX/Clustering}\\ % URL

\noindent This research was done under the supervision of Dr. Pauline Barmby with the financial support of the MITACS Globalink Research Internship Award within a total of 12 weeks, from June 16th to September 5th of 2014.\\ % License information

\noindent \textit{Версия 0.0.1 , Октябрь 2019} % Printing/edition date

%----------------------------------------------------------------------------------------
%	TABLE OF CONTENTS
%----------------------------------------------------------------------------------------

\chapterimage{cover_header.jpg} % Table of contents heading image

\pagestyle{empty} % No headers

\tableofcontents % Print the table of contents itself

%\cleardoublepage % Forces the first chapter to start on an odd page so it's on the right

\pagestyle{fancy} % Print headers again

%----------------------------------------------------------------------------------------
%	CHAPTER 1
%----------------------------------------------------------------------------------------

\chapterimage{cover_header.jpg} % Chapter heading image
\chapter{Convex Sets}
\chapter{Точные оценки параметров}
\section{Статистики и оценки}
Пусть $(\mathcal{X}, \mathcal{\beta}_{\mathcal{X}}, \mathcal{P})$ - вероятностное статистическая модель, где: 
\begin{remark}
$\mathcal{X} - $ множество всех возможных котиков,
$\mathcal{\beta}_\mathcal{X} - $  сигма алгебра над котиками,
$\mathcal{P} =  \{P_\theta : \theta \in \Theta\} - $  параметрическое семейство распределений,
$\Theta - $ множество параметров.
\end{remark}
Пусть $(X_1, \cdots, X_n) - $ выборка из неизвестного распределения $P \in \mathcal{P}$
\begin{defi}[Статистика]
Пусть $(E, \mathcal{E}) - $ измеримое пространство. Тогда измеримая функция $S:\mathcal{X} \rightarrow E$  называется статиской. \\
Если $E = \Theta$ то $S(x) - $ оценка $\theta$.
\end{defi}

\begin{exa}
Пусть $X = (X_1, \cdots , X_n) - $ дейсвительная выборка, т.е $\mathcal{X} = \mathrm{R}$ 
\begin{enumerate}
\item Выборочные характеристики: 
\begin{itemize}
\item $ \overline{g(X)}$ $ = \frac{1}{n} \sum_{i=1}^{n} g(X_i)$ - выборочная характеристика функции $g(X) - $ борелевская 
\item $\overline{X} = \frac{1}{n} \sum_{i=1}^n X_i - $  выборочное средние
\item $\overline{X^k} = \frac{1}{n}\sum_{i=1}^n X_i - k $ выборочный момент 
\end{itemize}
\item Функция от выборочных характеристик т.е функции вида $h(\overline{g_1(x)}, \cdots , \overline{g_k(x)})$
\begin{itemize}
\item $g_1(x) = x^2\,, g_2(x) = x\,, h(x,y) = x - y^2$\\ $h(\overline{g_1(x)},\overline{g_2(x)}) = h(\overline{X^2}, \overline{X}) = \overline{X^2} - \overline{X}^2 = S^2 - $ выборочная дисперсия 
\end{itemize}
\item Порядковые статистики 
\begin{itemize}
\item Упорядочим нашу выборку по возрастанию и обозначим её вот так: $$(X_{(1)}, \cdots , X_{(n)})$$ $X_{(k)} - $ $k$ порядковая статистика, $(X_{(1)}, \cdots , X_{(n)}) - $ называется вариационным рядом 
\item \textit{Численный пример.} Пусть у нас есть выборка $(X_1, X_2, X_3) = (2, 5, 1)$\\ $\overline{X} = \frac{2+5+1}{3} = \frac{8}{3}$ \\ $\overline{X^2} = \frac{2^2+ 5^2 + 1^2}{3} = 10$ \\ $S^2 = \overline{X^2} - \overline{X}^2 = 10 - (\frac{8}{3})^2 = \frac{26}{9}$ \\$(X_{(1)}, X_{(2)} , X_{(3)}) = (1,2,5)$
\end{itemize}
\end{enumerate}
\end{exa}
\section{Свойство оценок}
\begin{remark}
Для распределения $P_\theta$ будем обозначать так:
\begin{itemize}
\item $E_\theta - $ математическое ожидание
\item $D_\theta - $ Дисперсия
\item $P_\theta - $ почти наверное 
\item $d_\theta - $ сходимость по распределению
\end{itemize}
\end{remark}
\subsection{Не смещёность оценки }
\begin{defi}
Пусть $X = (X_1 , \cdots , X_2) $ выборка из неизвестного распределения  \\ $P \in \{P_\theta : \theta \in \Theta\}$, $\Theta \in \mathbf{R}^d$. Оценка $\hat{\theta}(x)$ называется не смещённое $\tau(\theta)$ если: $$E\hat{\theta}(X) = \tau(\theta)\text{,  } \forall \theta \in \Theta$$
\end{defi}
\begin{exa}
$\hat{\theta_1}(x) = \overline{X}$ , $\hat{\theta_1}(x) = X_1$ - несмещенные оценки для $\tau(\theta) = E_\theta X_1$
\begin{itemize}
\item $\mathcal{P} = \{\textit{Bern}(\theta) : \theta \in (0,1)\}$ , то $\overline{X}, X_1 - $ несмещенные оценки $\theta$ 
\item $\mathcal{P} = \{\textit{Exp}(\theta) : \theta > 0\}$, то $\overline{X}, X_1 - $  несмещённые оценки $1 / \theta$ 
\end{itemize}
\end{exa}
\subsection{Асимптотические свойства}
Пусть $X = (X_1, X_2,  \cdots) - $ выборка не ограниченного размера из распределения $P \in  \{P_\theta : \theta \in \Theta\} $

\begin{defi}
\label{def_as_property}
\\

\begin{enumerate}
\item Оценка $\hat{\theta}_n(X_1, \cdots, X_n)$ - называется состоятельной оценкой $\theta$ если: $$\hat{\theta}_n(X_1, \cdots, X_n) \xrightarrow[n \rightarrow \infty]{P_\theta} \theta \textbf{,  }\forall \theta \in \Theta $$
\item Оценка $\hat{\theta}_n(X_1, \cdots, X_n)$ называется сильно состоятельной оценкой $\theta$ если $$\hat{\theta}(X_1, \cdots, X_n) \xrightarrow[n \rightarrow \infty] {P_\theta - \textit{п.н}}  \theta ,\s \forall \t \in \T$$
\item Оценка $\H_n \X$ - называется асимптотической нормальной оценкой $\t$, если выполнено такое свойство: $$\sqrt{n}(\H_n \X - \t)  \DT \mathcal{N}(0,\Sigma) ,\s \forall \t \in \T $$ где $\Sigma(\t) - $ асимптотическая матрица ковариации, $\sigma^2(\t)$ - асимптотическая дисперсия.  
\end{enumerate}
\end{defi}
\begin{coro}
\begin{enumerate}
\item Состоятельность - при больших $n$ вероятность отклонения оценки $\HT$ от $\t$, нет численной характеристики степени отклонения 
\item Асимптатической нормальности  - числанная характеристика степени отклонения. \\ Пусть  $\HT - $ а.н.о. $\t$ с асимптатической дисперсией $\sigma^2(\t)$ $$\H \DT \mathcal{N}(\t, \sigma^2(\t) / n)$$ в многомерном случае $\Sigma(\t) - $ задаёт трубу в которой лежит оценка 
\end{enumerate}

\end{coro}

\begin{exa}
Пусть $\X$ - выборка из распределения Лапласа со сдвигом $\t$ плотностью $p_\t(x) = \frac{1}{2} e^{-|x-\t|}$ $,\s E_\t X_1= \t, \s D_\t = 2$\\ УЗБЧ: $\overline{X} \PPT \t \rightarrow - $ сильно состоятельная оценка $\t$  \\ ЦПТ: $\sqrt{n} (\overline{X} - \t) \DT \mathcal{N}(0,2)$ 
\end{exa}

\begin{proposition} Сильная состоятельность $\rightarrow$ Состоятельность, Асимптотическая нормальность $\rightarrow$ Состоятельность
\end{proposition}
\begin{proposition}
Пусть $\X - $ выборка такая что $E_\t |X_1|^{2k} < + \infty$. Тогда $\overline{X_k} - $ не смещенная, (сильно) состоятельная, асимптотическая нормальная оценка $E_\t X_1^k$ 
\end{proposition}

\section{Наследование свойств}
\textbf{Цель}: получить оценку для $\tau(\t)$, обладающие некоторым набором свойств, если имеестя оценка для  $\Psi(\t)$ с тем же свойствами. 

\begin{theorem}[О наследование сходимости]
\label{th_nasledovanoiy_shod}
Пусть $(\xi_n , n\in \mathbb{N})$, $\xi - $ случайный вектор размерности $d$, тогда если :
\begin{enumerate}
\item Если $\xi_n \PT \xi$ и $h: \mathbb{R}^d \rightarrow \mathbb{R}^k$ и $h$ непрерывная тогда $ h(\xi_n) \PT h(\xi)$ 
\item Аналогично для подчинаверной сходимости 
\item Если $\xi_n \d \xi$ и $h: \mathbb{R}^d \rightarrow \mathbb{R}^k - $ непрерывная то $h(\xi_n) \d h(\xi)$
\end{enumerate}
\end{theorem}
\begin{proof}[Доказательство \href{https://youtu.be/EL-V_0kWRoI?t=951}{Live}] 


\end{proof}

\begin{exa}
Пусть  $(\xi_n , n\in \mathbb{N}) - $ независимо одинаково распределённый случайные величины, такие что $E \xi_n = a \neq 0$, $D\xi_n < + \infty$ \\ ЗБЧ: $\frac{S_n}{n} \PT a, $ где $S_n = \xi_1 + \cdots + \xi_n$ \\ Рассмотрим $h(x) = 1/x$ и применим теорему: \\ $h(S_n/n) = \frac{n}{S_n} \PT \frac{1}{a} = h(a)$   
\end{exa}

\begin{proposition}
Пусть $\H -$ (сильно) состоятельная оценка $\t$ пусть $\tau$ непрерывна на $\T$. Тогда $\tau(\H) - $ (сильно) состоятельная оценка $\tau(\t)$ 
\end{proposition}

\begin{lemma}[\href{https://youtu.be/Ys7hMhGSnyE?t=5077} {Лемма Слуцкого}]
\label{lemma_sluc}
Пусть $(\xi_n , n\in \mathbb{N}) , (\eta_n , n\in \mathbb{N})$ $\xi - $ случайные величины, $c \in \mathbb{R}$. Пусть $\xi_n \d \xi$ , $\eta_n \d c$, тогда $\xi_n + \eta_n \d \xi + c$, $\xi_n \eta_n \d \xi c$
\end{lemma}
\begin{theorem}[\href{https://youtu.be/Ys7hMhGSnyE?t=5338}{О производной}]
\label{th_derivative}
Пусть$(\xi_n , n\in \mathbb{N}), \xi -$ случайный вектор размерности $d$ такие что $\xi_n \d \xi$ и $h:\mathbb{R}^d \rightarrow \mathbb{R}^k - $ непрерывна дифференцируема в точки $a \in \mathbb{R}^d$ и последовательность $b_n > 0, b_n \rightarrow 0$. \\ Тогда $\frac{h(a + \xi_nb_n) - h(a)}{b_n} \d \frac{\partial h}{\partial x}|_a \xi,$ где $\frac{\partial h}{ \partial x}|_a  - $ якобиан в точке $a$ 
\end{theorem}
\begin{proof}[$(d = 1):$]
Определим функцию $H(x) = \begin{cases} \frac{h(x+a) - h(a)}{x}, & \mbox{если } x \neq 0 \\ h'(a), & \mbox{если } x = 0\end{cases}$, эта функция непрерывна в точке $a$ давай   те применим лемму Слуцкого (Лемма: \ref{lemma_sluc}): $\xi_n b_n \d \xi 0 \underbrace{\Rightarrow}_{\includegraphics[scale=0.08]{SDKCG80ysUQ}} \xi_n b_n \p 0 $. \\
Дальше по теореме о наследование сходимости (Теорема: \ref{th_nasledovanoiy_shod}): 
$$H(\xi_n b_n) = \frac{h(\xi_n b_n+a) - h(a)}{\xi_n b_n} \p \d H(0) = h'(a) $$
Ёще раз применим лемму Слуцкого (Лемма: \ref{lemma_sluc}) (для этого мы подчеркнули то что если \textbf{есть сходимость по $\p$ то есть сходимость и по $\d$}, так же $\eta_n = H(\xi b_n), c = h'(a), \xi_n =  \xi_n$)

$$\underbrace{\xi_n H(\xi_n b_n)}_{\frac{h(a+\xi_n b_n) - h(a)}{b_n}} \d h'(a) \xi$$
\end{proof}
\begin{exa}[\href{https://youtu.be/Ys7hMhGSnyE?t=5927}{Live}]
Пусть $(\xi_n , n\in \mathbb{N}) - $ независимо случайно распределённые с.в. т.ч $E\xi_n = a \neq 0,\s D\xi_n = \sigma^2$, сходится ли следующее выражение (где $S = \xi_1 + \cdots + \xi_n$): 
$$\sqrt{n} \left(\frac{n}{S_n} - \frac{1}{a} \right) \d ?$$
\textbf{ Решение: }
Применяем ЦПТ: $$\sqrt{n} \left(\frac{S_n}{n} - a\right ) \d \mathcal{N}(0,\sigma^2)$$
Дальше воспользуемся теоремой о производной (Теорема: \ref{th_derivative}) с такими величинами  $\xi_n = \sqrt{n} \left(\frac{S_n}{n} - a \right ) $, $\xi \sim \mathcal{N}(0,\sigma^2), $ $h(x) = 1/x,\s b_n = 1/\sqrt{n} $: 
$$\frac{h(a  + b_n\xi_n) - h(a)}{b_n} = \sqrt{n} \left[h \left(a + \frac{1}{\sqrt{n}} \sqrt{n} \left(\frac{S_n}{n} - a \right ) \right) - h(a) \right] = \sqrt{n} \left[h \left(\frac{S_n}{n} \right ) - h(a)  \right] \rightarrow$$ $$\rightarrow \sqrt{n} \left[\frac{n}{s_n} - \frac{1}{a}\right] \rightarrow\textit{Теорема(\ref{th_derivative})} \d \xi \left. \left(\frac{1}{x} \right)' \right|_a = -\frac{1}{a^2} \xi \sim \mathcal{N}(0,\sigma^2 / a^4)$$
\end{exa}
\begin{claim}
Если мы расмотрим $\xi_n$ как выборку $\X$ то $1 /\overline{X} - $ это а.н.о. $1/a - $ с а.д. $\sigma^2/a^4$
\end{claim}
\begin{theorem}[\href{https://youtu.be/EL-V_0kWRoI?t=17}{Дельта Метод}] \label{th_delta_method}
Пусть $\H - $ аc. нормальная оценка $\t \in \T \subset \R^d$ с асимптотической матрицей ковариацией $\Sigma(\t)$, и пусть $\tau : \R^d \rightarrow \R^k - $непрерывно дифференцируема функция. \\ Тогда $\tau(\H) - $ ас. нормальная оценка для $\tau(\t)$ с ас. матрицей ковариацией $D(\t)\Sigma(\t)D(\theta)^T, $ где $\D(\t) = \frac{\partial \tau (\t)}{\partial \t}$
\end{theorem}
\begin{proof}
По определению асимптотической нормальной оценки (п. 3. Опр: \ref{def_as_property}). Если  $\H_n -$ ac. нормальная оценка тогда выполняется следующие: $$\sqrt{n}(\H_n - \t) \DT \N (0,\Sigma(\t))$$ Дальше применяем теорему о производной (Теорема: \ref{th_derivative}) с такими параметрами $a = \t, \s h(x) = \tau(x), \s \xi_n = \sqrt{n}(\H_n - \t), \s \xi \sim \N(0, \Sigma(\t)), \s b_n = 1/\sqrt{n} :$
$$\frac{h(a+\xi_nb_n) - h(a)}{b_n} = \frac{h(\t + \frac{1}{\sqrt{n}} \sqrt{n}(\H_n - \t)) - h(\t)}{1/\sqrt{n}} = \sqrt{n} \left[\tau(\H_n) - \tau(\t) \right] \rightarrow$$ $$\rightarrow \textit{Теорема о производной (\ref{th_derivative})} \DT \left. \underbrace{\frac{\partial h(\t)}{\partial \t}}_{D(\t)}\right|_{\t} \xi \sim \N(0, D(\t)\Sigma(\t) D(\t)^T)$$
\end{proof}

\begin{exa}[\href{https://youtu.be/EL-V_0kWRoI?t=504}{Live}] 
Пусть $\X \sim \Exp(\t)$. Тогда из ЦПТ мы зныем: $$\sqrt{n}(\overline{X} - 1/\t) \DT \N(0,1/\t^2) \Rightarrow \overline{X} - \textit{а.н.о. } 1/\t  \textit{ с а.д. } 1/\t^2$$ применяем Дельта Метод (Теорема \ref{th_delta_method}) с функцией $\tau(x) = 1/x:$ $$\tau(\overline{X}) = 1/\overline{X} - \textit{а.н.о. } \tau{(1/\t)} = \t \textit{ с асип. диспе.: } 1/\t^2 \left( \left.\frac{\partial \tau}{\partial x}\right|_{1/\t} \right)^2 = \t^2$$ 

\end{exa}





\end{document}